\section{Fragen zum Projekt}
\paragraph{Frage 1:} Wie behandeln Sie den doppelten Aufruf eines Prozesses solange dieser aktiv ist? (z.B. bei der Befehlsfolge: \textbf{b}, \textbf{b})
\paragraph{Antwort 1:} Generell wird ein doppelter Prozessaufruf nicht gesondert behandelt. Das bedeutet, dass bei erneutem Aufruf ein zus�tzlicher neuer Prozess mit eindeutiger PID angelegt wird. Danach teilt der Task-Verwalter den Prozessen in der �blichen Vorgehensweise die CPU-Zeit zu. \newline
Im Fall von \emph{Prozess a} kommt der reale doppelte Aufruf gar nicht vor, da sich der Prozess vor Ablauf der Zeitscheibe selbst beendet. Allerdings da jeder Prozess gleich behandelt wird, wird hier kein Unterschied gemacht.\newline
Beim doppelten Aufruf des Text-Prozess wird gepr�ft, ob der Prozess bereits existiert. Falls dies der Fall ist, wird er gel�scht. Dies ist der Befehlsstruktur des Prozesses geschuldet. 


\paragraph{Frage 2:} Wie reagiert ihr Programm auf die Benutzung der seriellen Schnittstelle 0 durch mehrere Prozesse?

\paragraph{Antwort 2:} Das Modul \emph{serial} verwendet das Hardwareflag \textbf{TI0} um die Verwendung der seriellen Schnittstelle zu kontrollieren. Vor dem Schreiben auf die serielle Schnittstelle muss gewartet werden, bis \textbf{TI0} gesetzt wurde. Ist das der Fall, wird \textbf{TI0} sofort auf 0 gesetzt und es kann geschrieben werden.\newline
Dadurch kann ein Prozess beim schreiben nicht von einem anderen Prozess unterbrochen werden. Das bedeutet allerdings auch, dass sich Prozesse blockieren k�nnen. Wollen zwei Prozesse gleichzeitig ein Zeichen ausgeben, muss der zweite Prozess so lange warten, bis der erste fertig mit schreiben ist. \newline
Durch dieses Verfahren wird zus�tzlich unterbunden, dass Zeichen "verschluckt" werden, da nur nach expliziter Freigabe der seriellen Schnittstelle geschrieben werden darf.


\paragraph{Frage 3:} Wie haben Sie die Anforderung Priorit�ten gel�st?

\paragraph{Antwort 3:}	
Jede Prozessart (\textbf{a}, \textbf{b}, \textbf{z}, \textbf{Konsole}) bekommt eine Zahl an Zeitscheiben zugeteilt. Die Anzahl spiegelt hierbei direkt die Priorit�t des Prozesses wieder. Nach Ablauf einer Zeitscheibe wird die Priorit�t dekrementiert. Ist die Priorit�t gleich 0 wird der Prozess gewechselt. Damit haben unterschiedliche Prozessarten immer unterschiedlich lang Rechenzeit von der CPU.
