%% Dokumentenklasse (Koma Script) -----------------------------------------
\documentclass[%
   draft,     % Entwurfsstadium
   final,      % fertiges Dokument
	 % --- Paper Settings ---
   paper=a4,% [Todo: add alternatives]
   paper=portrait, % landscape
   %papersize=auto, % driver
   % --- Base Font Size ---
   fontsize=12pt,%
   %%% scrpage2 options, need to be loaded with class or with scrpage2
   % headinclude,
   % headexclude,
   % footinclude,
   % footexclude,
	 % --- Koma Script Version ---
   version=last, %
	 % --- Math equation alignment ---
	 % reqno,
   % leqno,
%   fleqn, % Formeln linksb�ndig
 ]{scrbook} % Classes: scrartcl, scrreprt, scrbook

% Encoding der Dateien (sonst funktionieren Umlaute nicht)
% Fuer Linux -> utf8
% Fuer Windows, alte Linux Distributionen -> latin1

% Empfohlen latin1, da einige Pakete mit utf8 Zeichen nicht
% funktionieren, z.B: listings, soul.
\usepackage[latin1]{inputenc}
%\usepackage[ansinew]{inputenc}
%\usepackage[utf8]{inputenc}
%\usepackage{ucs}
%\usepackage[utf8x]{inputenc}

%%% Preambel
\input{preambel/settings}
\input{preambel/preambel}
\pdfminorversion=6		%erlaubt nun auch pdfs mit version 1.5
%
%%%% Neue Befehle
\input{macros/newcommands}
\input{macros/TableCommands}

%%% Silbentrennung
\hyphenation{
Brech-ungs-in-dex-un-ter-schied
Dia-mant-w\"ar-me-sprei-zer
Glas-pl\"att-chen
K\"uhl-e-le-ment
L\"an-gen-\"an-de-rung
o-ber-fl\"ach-en-e-mit-tier-en-den
Pel-tier-e-le-ment
Tran-si-en-ten
Wa-fer-po-si-tion
W\"ar-me-sprei-zer
W\"ar-me-spreiz-schicht
Ro-ta-ti-ons-ach-se
Feld-emis-si-on
Feld-emis-si-ons
Ras-ter-elek-tro-nen-mi-kro-skop
Ras-ter-elek-tro-nen-mi-kro-skops
nass-che-misch
nass-che-mische
Ver-un-rei-ni-gungen
}


%Zum Debuggen der Tikz-Bilder l�sst sich die preview-Umgebung nutzen
%plotall rendert nur alle Tikz-Bilder
%es kann auch nur ein Bildpfad eingegeben werden, so wird nur dieses eine Bild gerendert
%\DebugTikzpicture{plotall}
%\DebugTikzpicture{images/Fig4/Fig4}


%% Dokument Beginn %%%%%%%%%%%%%%%%%%%%%%%%%%%%%%%%%%%%%%%%%%%%%%%%%%%%%%%%
\begin{document}

%% - Deckblatt,
%% - Inhaltsverzeichnis,
%% - Hauptteil gegliedert z.B. in
%%   Einleitung, Grundlagen, Experimente, Ergebnisse, Zusammenfassung
%% - Anhang, (nicht mehr Bestandteil der Arbeit! Wird daher nicht bewertet)
%% - Literaturverzeichnis,
%% - Abbildungsverzeichnis (ggf.),
%% - Tabellenverzeichnis (ggf.),
%% - Abk�rzungsverzeichnis (ggf.),
%% - Formelverzeichnis (ggf.),
%% - Erkl�rung der Urheberschaft,


% Deckblatt
\pagenumbering{alph}

\begin{titlepage}
	
  \centering  
 	\rmfamily
 	
 	\vspace{4\baselineskip}
 	\normalsize \textbf{Programmentwurf}\\
 	\normalsize \textbf{Sytemnaheprogrammierung}
 	\vspace{2\baselineskip}\\
 	\huge \textbf{Task-Verwaltung}
 	\vspace{3\baselineskip}\\
 	\large \textsc{Stanislav Sokol\\Dominik Zipperle}
 	\vspace{1\baselineskip}\\
 	\large \today
  \vspace{5\baselineskip}\\
  \large
  	\begin{center}
  		Kurs TINF13IN\vspace{1\baselineskip}\\4. Semester\vspace{1\baselineskip}\\2015
  	\end{center}
    

\end{titlepage}


% Inhaltsverzeichnis
\frontmatter
\pagenumbering{Alph}
\begingroup
  \renewcommand*{\chapterpagestyle}{empty}
  \pagestyle{empty}
  \tableofcontents
  \clearpage
\endgroup


% Hauptteil
%\mainmatter
\pagenumbering{arabic}

\section{Problembeschreibung}
Das im Folgenden beschriebene Programm befasst sich mit der Prozessverwaltung auf dem 8051. Dies ist notwendig, damit mehrere Prozesse pseudoparallel auf einer CPU arbeiten k�nnen. Hierbei soll eine Art Kontext-Switch Strategie entwickelt werden. Somit sieht es f�r jeden Prozess so aus, als w�re dieser allein im System. Die Prozessverwaltung umfasst das Erzeugen und L�schen eines Prozesses, sowie die Umschaltung zwischen den Prozessen im Speicher. 

F�r den Funktionstest werden die in der Anforderung beschriebenen Prozesse implementiert und getestet.  
\section{Fragen zum Projekt}
\paragraph{Frage 1:} Wie behandeln Sie den doppelten Aufruf eines Prozesses solange dieser aktiv ist? (z.B. bei der Befehlsfolge: \textbf{b}, \textbf{b})
\paragraph{Antwort 1:} Generell wird ein doppelter Prozessaufruf nicht gesondert behandelt. Das bedeutet, dass bei erneutem Aufruf ein zus�tzlicher neuer Prozess mit eindeutiger PID angelegt wird. Danach teilt der Task-Verwalter den Prozessen in der �blichen Vorgehensweise die CPU-Zeit zu. \newline
Im Fall von \emph{Prozess a} kommt der reale doppelte Aufruf gar nicht vor, da sich der Prozess vor Ablauf der Zeitscheibe selbst beendet. Theoretisch wird der Prozess aber nicht anders als die anderen Prozesse gehandhabt.\newline
Beim doppelten Aufruf des Text-Prozess wird gepr�ft, ob der Prozess bereits existiert. Falls dies der Fall ist, wird er gel�scht. Dies entspricht der vorgegeben Befehlsstruktur des Prozesses.


\paragraph{Frage 2:} Wie reagiert ihr Programm auf die Benutzung der seriellen Schnittstelle 0 durch mehrere Prozesse?

\paragraph{Antwort 2:} Das Modul \emph{serial} verwendet das Hardwareflag \textbf{TI0} um die Verwendung der seriellen Schnittstelle zu kontrollieren. Vor dem Schreiben auf die serielle Schnittstelle muss gewartet werden, bis \textbf{TI0} gesetzt wurde. Ist das der Fall, wird \textbf{TI0} sofort auf 0 gesetzt und es kann geschrieben werden.\newline
Dadurch kann ein Prozess beim schreiben nicht von einem anderen Prozess unterbrochen werden. Das bedeutet allerdings auch, dass sich Prozesse blockieren k�nnen. Wollen zwei Prozesse gleichzeitig ein Zeichen ausgeben, muss der zweite Prozess so lange warten, bis der erste fertig mit schreiben ist. \newline
Durch dieses Verfahren wird zus�tzlich unterbunden, dass Zeichen "verschluckt" werden, da nur nach expliziter Freigabe der seriellen Schnittstelle geschrieben werden darf.

\newpage

\paragraph{Frage 3:} Wie haben Sie die Anforderung Priorit�ten gel�st?

\paragraph{Antwort 3:}	
Jede Prozessart (\textbf{a}, \textbf{b}, \textbf{z}, \textbf{Konsole}) bekommt eine feste Zahl an Zeitscheiben zugeteilt. Die Anzahl spiegelt hierbei direkt die Priorit�t des Prozesses wieder. Nach Ablauf einer Zeitscheibe wird die Priorit�t dekrementiert. Ist die Priorit�t gleich 0, wird der Prozess gewechselt. Damit haben unterschiedliche Prozessarten immer unterschiedlich lang Rechenzeit von der CPU.

\section{Kontrollausgaben}
Im Folgenden werden die Kontrollausgaben aus den vorgebenden Befehlssequenzen erzeugt.


\subsection{Sequenz 1 aus os1.inc}

\begin{align*}
\text{Befehlssequenz} &: \mathtt{x,a,b,x,c,a,b,a,c,b,b,a,c,c}
\end{align*}
Kontrollausgabe:
\lstinputlisting{./listing/os1.dat}




\subsection{Sequenz 2 aus os2.inc}
\begin{align*}
\text{Befehlssequenz} &: \mathtt{a,z,z,a,z,z,a}
\end{align*}

Kontrollausgabe:
\lstinputlisting{./listing/os2.dat}


\subsection{Sequenz 3 aus os3.inc}

\begin{align*}
\text{Befehlssequenz} &: \mathtt{z,b,a,z,z,a,c,z,a}
\end{align*}

Kontrollausgabe:
\lstinputlisting{./listing/os3.dat}
\newpage
\section{Beschreibung der Module}
	\subsection{Kurbeschreibung}
	Das gesamte Projekt besteht auf folgenden Modulen (Flussdiagramme sind im Kapitel~\ref{Module}):
	\begin{description}[font=\sffamily\bfseries, leftmargin=1.5cm,style=sameline] 
	\item{\textbf{main}}
	Dies ist das Hauptprogramm. Beim Start werden die serielle Schnittstelle und das \textbf{scheduler}-Modul initialisiert. Das Konsolenprozess wird erzeugt. Timer-Interrupts werden initialisiert. Danach befindet sich das Hauptprogramm in einer Endlosschleife.
	
	\item{\textbf{proc\_console}}
	Dies ist der Konsolenprozess. Der Prozess ist eine Endlosschleife, welche die Eingaben von der seriellen Schnittstelle abholt. Diese Eingaben werden auf �bereinstimmung mit den bekannten Befehlen gepr�ft. Unbekannte Eingaben werden verworfen. Befehle werden erkannt und ausgef�hrt.
	\item{\textbf{scheduler}} Dies ist der eigentliche Task-Verwalter. Das Modul besteht aus Subroutinen:
	\begin{description}[font=\sffamily\bfseries, leftmargin=0.5cm,style=sameline] 
	\item{\textbf{new\_proc}}
		Hier wird ein neuer Prozess erzeugt. In der Prozesstabelle im externen RAM wird nach dem ersten freien Platz gesucht (first fit). Jeder Prozess wird vereinfacht gleich gro� angenommen.
		Die Position des gefundenen freien Platzes in der Prozesstabelle entspricht einer Position vom Datenbereich der Prozesse. Die Datenbereiche werden initialisiert.
		
	\item{\textbf{del\_proc}} Hier wird der Verweis auf den Datenbereich eines zu beendenden Prozesses aus der Prozesstabelle entfernt. Hiermit kennt der Task-Verwalter diesen Prozess und seinen Datenbereich nicht mehr.
	
	\item{\textbf{change\_proc}}
	Hier wird der Kontext der Prozesse getauscht. Hierbei werden alle prozessrelevanten Daten in den externen Speicher ausgelagert (�hnlich wie Swapping). N�chster Prozess wird aus der Prozesstabelle geholt und sein Kontext wird ins RAM geladen.
	
	\end{description} 
	
	\item{\textbf{serial}} Dieses Modul stellt Routinen zum I/O auf der seriellen Schnittstelle bereit.
	
	\item{\textbf{proc\_a}} Prozess a: Eine Zeichenfolge \textquotedblleft abcde\textquotedblright wird auf der seriellen Schnittstelle 0 ausgegeben.
	\item{\textbf{proc\_b}} Prozess b: Gibt jede Sekunde \textquotedblleft+\textquotedblright auf die serielle Schnittstelle aus
	\item{\textbf{proc\_z}} Prozess z: Ruft fkt\_text auf
	\item{\textbf{fkt\_text}} Testprozess: gibt eine Zeichenfolge auf serielle Schnittstelle 1 aus.
	\end{description}
	
\newpage
\subsection{Speichernutzung und Variablen}

F�r das Programm werden folgende Speicherbereiche reserviert und verwendet:

\subsubsection*{Variablen und Konstanten}

Variablen und Konstanten werden im \textquotedblleft variables.inc\textquotedblright{} dem Befehl \begin{center}
<NAME> EQU <HEX>
\end{center} angelegt. Durch einbinden der Datei in jedes Modul sind die Variablen global bekannt. Die genaue Beschreibung der Variablen und deren Funktion ist einem  kommentierten Listing im Anhang zu entnehmen.

\subsubsection{Externer Datenbereich}
Externer Datenbereich wird folgenderma�en genutzt. Zu Beginn des Programms werden zwei Datenbereiche im externen Speicher reserviert. Ein Bereich ist die Prozesstabelle. Pro Prozess sind hier 4 Byte vergeben. Die setzen sich wie folgt zusammen. 2 Byte DPTR auf den dazugeh�rigen Prozessdatenbereich, 1 Byte Bitmaske mit Prozessstatus und Prozesstyp und 1 Byte f�r die Prozess ID.

Es sind insgesamt 20 Prozesse m�glich. Die Entscheidung f�r die Zahl 20 ist willk�rlich und kann auch generell auf \textquotedblleft beliebig\textquotedblright{} gesetzt werden. Die Grenze bildet hier die Gr��e des externen Datenbereichs. 

Jeder Prozess besitzt seinen eigenen Datenbereich, worin sein Kontext aufbewahrt wird. Pro Prozess sind 32 Byte reserviert. Aktuell werden lediglich 23 Byte benutzt. 10 Byte f�r den Stack und 13 Byte f�r den geforderten Kontext. Eine Erweiterung der Stackgr��e ist m�glich. Abbildung~\ref{fig:ext_data} zeigt die Speicherbelegung. 
\begin{figure}[H]
\centering
\includegraphics[width=0.7\textwidth]{./images/ext_data}
\caption{Speicherbelegung des externen RAM}
\label{fig:ext_data}
\end{figure}

Die Abbildung~\ref{fig:dataprocess} zeigt die reale Speicherbelegung eines eingelagerten Prozesses. Dabei sind die kleineren Adressen unten.

\begin{figure}[H]
\centering
\includegraphics[width=0.7\textwidth]{./images/data_process}
\caption{Speicherbelegung des externen RAM f�r einen Prozess}
\label{fig:dataprocess}
\end{figure}
\newpage
\subsection{Module\label{Module}}
\subsubsection{Hauptmodul \texttt{main.a51}}
In der Abbildung~\ref{fig:main} ist das Ablaufdiagramm vom Hauptmodul dargestellt. Die genaue Beschreibung der einzelnen Operationen ist aus dem Listing im Anhang zu entnehmen.

\begin{figure}[H]
\centering
\includegraphics[width=0.5\textwidth]{./images/main}
\caption{Ablaufdiagramm main.a51}
\label{fig:main}
\end{figure}

\subsubsection{Konsolenprozess \texttt{proc\_console.a51}}
In der Abbildung~\ref{fig:console} ist das Ablaufdiagramm vom Hauptmodul dargestellt. Der Konsolenprozess ist eine Endlosschleife. Die genaue Beschreibung der einzelnen Operationen ist aus dem Listing im Anhang zu entnehmen.

Bemerkung: Das Vorbereiten des Prozesses, also das setzen der Parameter ist ein eigenst�ndiger Code-Block. Wegen der �bersicht wird dies als Funktion dargestellt.

\begin{figure}[H]
\centering
\includegraphics[width=\textwidth]{./images/console}
\caption{Ablaufdiagramm \texttt{proc\_console.a51}}
\label{fig:console}
\end{figure}
\newpage

\subsubsection{Taskverwalter \texttt{scheduler.a51}}

\paragraph{new\_process} In der Abbildung~\ref{fig:newproc} ist das Ablaufdiagramm f�r die Erstellung eines neuen Prozesses dargestellt. Dies ist ein Subprozess des Taskverwalter Moduls. Hierbei werden in die 4 Byte der Prozesstabelle die n�tigen Informationen eingetragen [Startadresse des Prozesses (2 Byte), Bitmaske der Priorit�t und der Prozesstyp ID (1 Byte), Prozess ID (1 Byte) ].


So setzt sich die Bitmaske des 3 Bytes eines Prozess in der Prozesstabelle:

\begin{align*}
B\underbrace{\si{xxxxx}}_{\mathtt{Priorit�t}}\underbrace{\si{xxx}}_{\mathtt{Prozesstyp}}
\end{align*}

Die Prozess ID ist ein HEX-Wert welcher der aktuellen Position des Prozesses in der Tabelle entspricht.

\begin{figure}[H]
\centering
\includegraphics[width=\textwidth]{./images/new_proc}
\caption{Ablaufdiagramm \texttt{newproc}}
\label{fig:newproc}
\end{figure}

\paragraph{delete process}
In der Abbildung~\ref{fig:delproc} ist das Ablaufdiagramm des Subprozesses del\_proc dargestellt. Hierbei werden zun�chst die Daten der aktuellen Routine gerettet. Die Prozesstyp ID wird aus der Bitmaske im 3 Byte der Prozesstabelle extrahiert und mit dem zu l�schenden Typ verglichen. Bei �bereinstimmung wird die Referenz auf den Prozessdatenbereich des zu l�schenden Prozesses entfernt.

Bemerkung: Hier ist es wichtig zu erw�hnen, dadurch dass Prozess b in mehreren Instanzen vorkommen kann und er lediglich durch ein einziges Kommando c gel�scht werden kann, wird der erst beste Prozess vom Typ b auf Kommando gel�scht. Um Prozess gezielt zu entfernen m�sste man das Beenden Kommando anpassen.

\begin{figure}[H]
\centering
\includegraphics[width=\textwidth]{./images/del_proc}
\caption{Ablaufdiagramm \texttt{delete process}}
\label{fig:delproc}
\end{figure}

\paragraph{change process}

In der Abbildung~\ref{fig:changeproc} ist das Ablaufdiagramm des Subprozesses change process dargestellt. Dieser Subprozess sucht anhand der aktuellen Position des Prozesses den n�chsten in der Prozesstabelle. Da es durchaus leere Eintr�ge geben kann, werden diese �bersprungen. Change process wird nur aufgerufen, wenn ein Prozess seine Zeitscheibe bei der CPU aufgebraucht hat. 

Die Daten des aktuellen Prozesses werden in den externen Prozessdatenbereich ausgelagert und die Daten des neuen Prozesses werden eingeladen. Die Daten werden im Kapitel Speichernutzung erkl�rt.

\begin{figure}[H]
\centering
\includegraphics[width=\textwidth]{./images/change_proc}
\caption{Ablaufdiagramm \texttt{change process}}
\label{fig:changeproc}
\end{figure}

\paragraph{save and restore data}
In der Abbildung~\ref{fig:restsavproc} ist die Subroutine zum speichern und retten des Kontexts gezeigt. Die Auflistung der Daten ist dem Kapitel Speichernutzung zu entnehmen.

\begin{figure}[H]
\centering
\includegraphics[width=0.6\textwidth]{./images/restore_save_proc}
\caption{Ablaufdiagramm \texttt{restore and save process}}
\label{fig:restsavproc}
\end{figure}

\subsubsection{Prozess a}

In der Abbildung~\ref{fig:proca} ist die Implementierung des Prozess A dargestellt. Prozess A gibt eine Zeichenfolge aus und begibt sich in eine Endlosschleife bis er beendet wird.

		\begin{figure}[H]
		\centering
		\includegraphics[width=0.3\textwidth]{./images/proc_a}
		\caption{Ablaufdiagramm des Prozess a}
		\label{fig:proca}
		\end{figure}
		
		\subsubsection{Prozess b}
		
		In der Abbildung~\ref{fig:procb} ist die Implementierung des Prozess B dargestellt. Beim Start holt sich der Prozess den aktuellen Wert des Sekundenz�hlers (ganze Sekunden). In einer Schleife wird gepr�ft, ob sich der Systemwert des Z�hlers ver�ndert hat. Hiermit ist eine Sekunde vergangen. Danach wird + Zeichen auf die Konsole ausgegeben. 
		
				\begin{figure}[H]
				\centering
				\includegraphics[width=0.5\textwidth]{./images/proc_b}
				\caption{Ablaufdiagramm des Prozess b}
				\label{fig:procb}
				\end{figure}

\newpage




% Anhang
\appendix
\pagenumbering{Roman}
\section{Listings}
\subsection{Main}
\lstinputlisting{./listing/main.a51}
\subsection{Scheduler}
\lstinputlisting{./listing/scheduler.a51}
\subsection{Serielle Schnittstelle}
\lstinputlisting{./listing/serial.a51}
\subsection{Konsolenprozess}
\lstinputlisting{./listing/proc_console.a51}
\subsection{Prozess A}
\lstinputlisting{./listing/proc_a.a51}
\subsection{Prozess B}
\lstinputlisting{./listing/proc_b.a51}
\subsection{Prozess Z}
\lstinputlisting{./listing/proc_z.a51}



% Verzeichnisse
%\backmatter
\pagenumbering{roman}
\renewcommand{\thetable}{C.\arabic{table}}1



%% Dokument ENDE %%%%%%%%%%%%%%%%%%%%%%%%%%%%%%%%%%%%%%%%%%%%%%%%%%%%%%%%%%
\end{document}


\immediate\closeout\Imagelist

